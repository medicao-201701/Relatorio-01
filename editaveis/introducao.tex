\chapter{Introdução}

\section{Contexto}
	
	A pesquisa de medição é abordada no contexto da disciplina de Introdução a Jogos Eletrônicos, mais especifícamente com o projeto Elemental Boss que está sendo desenvolvido por sete alunos tanto da Engenharia de Software, como dos cursos de Música e de Desenho Industrial. Os alunos devem desenvolver um jogo eletrônico a partir do zero, ou seja, eles devem criar desde a engine até todas as artes, músicas e o level design do mesmo. Logo no âmbito da engenharia de software, eles precisam fazer decisões arquiteturais e de padrões de projeto.
	Como o desenvolvimento de jogos é geralmente recheado de detalhes, muitas vezes as questões de medição e análise de projeto não são levadas em conta, principalmente em equipes de porte pequeno, podendo causar em diversas falhas e até mesmo no cancelamento do projeto. Logo o objetivo da equipe de medição é de aplicar estratégias para tentar assegurar um bom desenvolvimento.

\section{Formulação do Problema}

	O projeto de Introdução a Jogos Eletrônicos é divido em vários marcos, na qual cada marco está relacionado com uma parte do desenvolvimento do jogo. Nosso foco é então adotar estratégias que façam com que haja sempre um constante desenvolvimento de todos os marcos do projeto, assim reduzindo bastante o risco do cancelamento do projeto e ajudando também na manutenibilidade do mesmo.

\section{Objetivos da organização}

\subsection{Objetivo geral}
	O objetivo desse trabalho é tornar melhor a experiência do grupo que está desenvolvendo o jogo proposto, de forma que com o fornecimento de dados, informações e as devidas interpretações dos mesmos, o processo de desenvolvimento evolua, fazendo com que um produto de melhor qualidade seja produzido. Tudo isso depende da escolha correta de quais informações o grupo vai coletar e analisar, a fim de que o time de desenvolvimento obtenha sucesso na disciplina de Introdução aos Jogos Eletrônicos.

\subsection{Objetivos específicos}


\section{Justificativas}
  
