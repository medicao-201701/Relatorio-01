\chapter{Fundamentação Teórica}

Resumidamente, esse trabalho consiste em estudos e aplicacões de métricas em um contexto individual na área acadêmica de Introducão de Jogos Eletrônicos, visando o ganho de conhecimento acerca de Medicão e Análise de software, e também com intuito de garantir uma melhor entrega do jogo a ser produzido.
Adquirir conhecimentos de medicão e análise tanto dentro quanto fora da sala de aula, envolvendo muita leitura e pesquisa. Mas também é de suma importância conhecer o processo em que os alunos de Introducão de Jogos Eletrônicos são submetidos a seguirem para a entrega do produto.
As bibliografias que mais utilizaremos serão:
\begin{itemize}
\item Goal Question Metric Paradigm [Victor R. Basili, H. Dieter Rombach]
\item Metrics and Models in Software Quality Engineering [Stephen H. Kan]
\end{itemize}