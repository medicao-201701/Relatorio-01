\chapter{Metodologia}

\section{Processo de medição}

Com o objetivo de garantir que os dados coletados pela equipe de medição sejam úteis, a equipe usará de um processo baseado na ISO/IEC/IEEE 15939:2008 apresentada no SWEBOK no capítulo de gerência de projetos. Nem todas atividades foram contempladas, apenas aquelas consideradas de maior valor para o projeto a ser desenvolvido. Esse processo segue o ciclo PDCA - Plan, do, check, act (planejar, fazer, verificar, agir) e está sendo ilustrado na figura abaixo:

\begin{figure}[!htpb]
\centering
\includegraphics[scale=0.4]{figuras/processo/processo}
\caption{Processo de medição a ser utilizado pela equipe}
\end{figure}


O processo elaborado também seguirá o modelo MPS-BR do Processo de medição, presente no nível F - Gerenciado. Segundo o MPS, o propósito do processo de medição nesse nível é coletar, armazenar, analisar e relatar dados relativos ao produto desenvolvido e processo de desenvolvimento da organização. Logo abaixo as atividades do processo elaborado estão mapeadas de acordo com cada tópico do processo MPS-BR:

\begin{itemize}
\item MED 1 - Objetivos   de  medição   são   estabelecidos   e  mantidos  a  partir   dos 
objetivos   de   negócio   da   organização   e   das   necessidades   de informação de processos técnicos e gerenciais: Definir Escopo, Definir papéis, Caracterizar unidade organizacional, Identificar necessidades de informação
\item MED 2 - Um  conjunto  adequado  de  medidas,  orientado  pelos  objetivos  de medição,  é  identificado  e  definido,  priorizado,  documentado,  revisado 
e, quando pertinente, atualizado: Levantar questões sobre os objetivos, Levantar métricas para as questões, Validar GQM
\item MED 3 - Os  procedimentos  para  a  coleta  e  o  armazenamento  de  medidas  são especificados: Levantar métricas para as questões,  Definir tecnologias de apoio
\item MED 4 - Os procedimentos para a análise das medidas são especificados: Levantar métricas para as questões, Definir tecnologias de apoio
\item MED 5 - Os dados requeridos são coletados e analisados: Coletar Dados, Analisar os Dados, Avaliar o processo de execução
\item MED 6 - Os dados e os resultados das análises são armazenados: Coletar Dados, Analisar os Dados
\item MED 7 - Os   dados   e   os   resultados   das   análises   são   comunicados   aos interessados e são utilizados para apoiar decisões: Comunicar resultados, Comunicar melhorias propostas

\end{itemize}